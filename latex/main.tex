\documentclass[12pt,english,letterpaper]{kuthesis}
\usepackage[T1]{fontenc}
\usepackage[utf8]{inputenc}
\usepackage{mathptmx}
\renewcommand{\sfdefault}{lmss}
\renewcommand{\ttdefault}{lmtt}
\usepackage{geometry}
\geometry{verbose,tmargin=1in,bmargin=1in,lmargin=1in,rmargin=1in}
\setcounter{secnumdepth}{3}
\setcounter{tocdepth}{3}
\usepackage{color}
\usepackage{babel}
\usepackage{url}
\usepackage{subcaption}
\usepackage{graphicx}
\usepackage{float}
\usepackage{setspace}
\usepackage{import}
\usepackage[authoryear]{natbib}
\doublespacing
\usepackage[unicode=true,
 bookmarks=true,bookmarksnumbered=false,bookmarksopen=false,
 breaklinks=true,pdfborder={0 0 0},pdfborderstyle={},backref=false,colorlinks=true]
 {hyperref}
\hypersetup{pdftitle={University of Kansas Thesis Template},
 pdfauthor={Anonymous},
 pdfsubject={A Thesis},
 urlcolor={black},citecolor={black},allcolors={black}}
\makeatletter
% used to align decimals in tables according to APA style
\usepackage{dcolumn}
\usepackage{booktabs}
% Don't change this block, probably.
% This handles code input. Included here just for completion.
\usepackage{listings}

%%%%%%%%%%%%%%%%%%%%%%%%%%%%%%%%%%%%%%%%%%%%%%%%%%%%%%%%%%%%%%%%%%%%%%%%%%%%%%%%%%%%%%%%%
% This is the stuff you should edit.
%%%%%%%%%%%%%%%%%%%%%%%%%%%%%%%%%%%%%%%%%%%%%%%%%%%%%%%%%%%%%%%%%%%%%%%%%%%%%%%%%%%%%%%%%
\title{Some Title}
\author{Mingyoung Jeng}
% Optional priorcreds field
% Leave blank if you don't want to list prior credits
% \priorcreds{B.M. Caffeine}{B.S. in Caffeine}
\priorcreds{B.S. in Engineering Physics}
\dept{Department of Electrical Engineering and Computer Science}
\degreetitle{Master's of Computer Engineering}
\papertype{Thesis} % or Dissertation 
%  It is required to have 7 entries, even if some are empty, i.e., {}  for committee and role
\committee{Member Name 1}{Member Name 2}{Member Name 3}{Member Name 4}{Member 5 has a long name}{Member 6}{}
\role{Chairperson}{Co-Chair}{Some description}{No Role}{}{External Reviewer}{}
\@printd@testrue
% BOTH dates must be included. 
\datedefended{July 02, 2019}
 % Since the date will likely be unknown, this keeps the blank line from floating above the "Date Approved: " line late signing.
\dateapproved{\textcolor{white}{blank}}
%%%%%%%%%%%%%%%%%%%%%%%%%%%%%%%%%%%%%%%%%%%%%%%%%%%%%%%%%%%%%%%%%%%%%%%%%%%%%%%%%%%%%%%%%
% After \makeatother, you put the packages you need. 
%%%%%%%%%%%%%%%%%%%%%%%%%%%%%%%%%%%%%%%%%%%%%%%%%%%%%%%%%%%%%%%%%%%%%%%%%%%%%%%%%%%%%%%%%

% This is definitely a hack, and I know what it does, but not how to put it somewhere else. This likely has to do with how the .cls file is handled, which I did not create. -AB
\makeatother

% Individual user defined packages can go here.
\usepackage{subcaption}
\usepackage{physics}

\begin{document}

% This is the table of contents, list of figures, list of tables, formatted correctly.
% romanpages ~ pages numbered by roman numerals
\begin{romanpages}
	\maketitle
	
	\begin{abstract}
		TBD
	\end{abstract}

	\begin{acknowledgementslong}
		Acknowledgements go here.
	\end{acknowledgementslong}

	\tableofcontents{}
	\listoffigures
	\listoftables
\end{romanpages}

\chapter{Introduction}

\section{Motivation}



\section{Problem Statement}
\section{idk}
\chapter{Background}

\section{Fundamental Concepts}
On the macroscopic scale, objects and events are largely expected to have a defined state. [examples], and classical bit exists in either a \texttt{0} or \texttt{1} state.
In general, however, there is a limit to the amount of information that can be gathered with certainty, as dictated by the Heisenberg uncertainty principle, see \eqref{eq:heisenberg_uncertainty}.

\begin{equation}
	\sigma_x \sigma_p \geq \frac{\hbar}{2},
\label{eq:heisenberg_uncertainty}
\end{equation}
where $\sigma_x$ and $\sigma_p$ represent the standard deviation in position and momentum, respectively, and $\hbar$ is the reduced Planck constant. In other words, there must always exist some uncertainty in knowing a particle's position and/or momentum, although it is small enough such that it typically only manifests at quantum scales.

Thus, in order to leverage classical physics, quantum states are generalized in terms of probabilistic wavefunctions, representing the weighted complex probability amplitudes spanning the entire state space of a quantum object. As the name implies, wavefunctions behave like waves, which can be interpreted as [...].

What is a Hilbert space - ``a vector space equipped with an inner product that defines a distance function for which the space is a complete metric space.''

The Dirac or ``bra-ket'' notation...



% Wave behavior / probability interpretation



% \section{test}
% discrete quantized states
% linear vector space, hilbert space
% braket / Dirac notation



% Classical Convolution
% Quantum Wavefunctions
% Quantum Gates
% Multidimensional Quantum Amplitude Encoding

% Related Work

\chapter{Methods}

Representation of Filter (as a data encoding problem)
$\ketbra{\psi}{\psi}$

Convolutional Layers

Pooling Layers

Fully-connected Layer

\chapter{Results and Analysis}
\chapter{Conclusion}

% Bibliography, default style from former template.
\global\long\def\bibname{References}
\bibliographystyle{unsrt}
\bibliography{refs}

% \appendix

\end{document}
